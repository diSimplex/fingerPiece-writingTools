% !TEX root = writingTools.tex
% !LPiL preamble = ./wtPreamble.tex
% !LPiL postamble = ./wtPostamble.tex

\lpilTitle{fp-writing}{
  Finger Pieces : Writing Tools
}
\author{Stephen Gaito}

\begin{abstract}
  In this finger piece, we explore the variouse tools we use to write
  diSimplex/LPiL documents.
\end{abstract}

\maketitle

The diSimplex project as a whole is a very complex task, involving weaving
together the research of many individuals across a wide range of disciplines.

The aim of the diSimplex project is to understand how a finite
\emph{computational} agent can \emph{model} reality. In particular what is the
\emph{computational} basis for Mathematical Physics. Along the way we
\emph{hope} to provide a foundation for Quantum Relativity.

This means that various parts of the overall project are going to require that
we write, document, and prove corrent runable code. To describe this code as
well as a foundation for Mathematical Physics, we will need a range of diagrams.

To manage this complexity we use a number of tools, both
`standard-off-the-shelf' as well as a number of bespoke additions where needed.
This `finger-piece' attempts to document what these tools are and how they are
used.

\textbf{Editor} : We use the VSCode/VSCodium editing environment
together with James Yu's excellent Latex-Workshop.

\textbf{Typesetters} : We use \emph{both} the LaTeX \emph{and} the
ConTeXt typesetting tools.
\begin{itemize}
  \item We use \textbf{standard LaTeX} for the bulk of the writing for two
  reasons:
  \begin{enumerate}
    \item Unfortunately almost no journals currently accept written
    submissions which use ConTeXt,
    \item The tools available for writing ConTeXt documents, certainly using
    VSCode, are not as well developed as those for writing using LaTeX.
  \end{enumerate}
  \item We use \textbf{ConTeXt's excellent MetaFun diagram package} (based
  upon MetaPost), for creating diagrams.
\end{itemize}

\textbf{Integration (VSCode)} : We use our own \verb|lpilMagicRunner| to help
the LaTeX-Workshop typeset the LaTeX writing, the ConTeXt/MetaFun diagrams as
well as pygmentize the various chunks of code.

\textbf{Integration (whole project)} : We use our own \verb|cfdoit| to manage
the `compilation' of a whole document (including compiling and verifying the
code).

% !TEX root = writingTools.tex
% !LPiL preamble = ./wtPreamble.tex
% !LPiL postamble = ./wtPostamble.tex

\lpilSection{fp-wt-vscode}{Using VSCode/VSCodium}

To use VSCode (or VSCodium):

\begin{enumerate}
  \item Install \href{https://code.visualstudio.com/}{VSCode} (or
  \href{https://vscodium.com/}{VSCodium})
  \item Install the
  \href{https://open-vsx.org/extension/James-Yu/latex-workshop}{VSCode
  LaTeX-Workshop extension}
  \item Using \href{https://code.visualstudio.com/docs/getstarted/settings}{the
  VSCode settings interface}
  \begin{enumerate}
    \item Set \verb|latex-workshop.latex.outdir| to \verb|build/latex|
    \item Set \verb|latex-workshop.latex.tools| to 
\begin{lpil:json}{latexTools.json}
  [
    {
      "args": [
        "%DOC%",
        "%OUTDIR%"
      ],
      "command": "lpilMagicRunner",
      "env": {},
      "name": "lpilMagicRunner"
    }
  ]
\end{lpil:json}
    \item Set \verb|latex-workshop.latex.recipes| to
\begin{lpil:json}{latexRecipes.json}
  [
    {
      "name": "lpilMagicRunner",
      "tools": [
        "lpilMagicRunner"
      ]
    }
  ]
\end{lpil:json}
  \end{enumerate}
\end{enumerate}

% !TEX root = writingTools.tex
% !LPiL preamble = ./wtPreamble.tex
% !LPiL postamble = ./wtPostamble.tex

\lpilSection{fp-wt-typesetters}{Typesetters}

To install LaTeX and ConTeXt, we use the excellent
\href{https://tug.org/texlive/}{texlive distribution}. You can either install
the `full' distribution `scheme`, or one or more of its various `collections'
(see the
\href{https://tug.org/texlive/doc/texlive-en/texlive-en.html#x1-240003.2.2}{Selecting
what is to be installed} section of the
\href{https://tug.org/texlive/doc.html}{texlive documentation}). 

At a bare minium you will need to ensure \verb|lualatex| and \verb|context| are
installed. (This \emph{should} also ensure \verb|metapost| is installed as well).

We use our own \href{https://github.com/litProgLaTeX}{`Literate Programming in
LaTeX' tools}, \verb|lpilMagicRunner|, \verb|cfdoit| as well as the
\href{}{lpil-latex} style.

% !TEX root = writingTools.tex
% !LPiL preamble = ./wtPreamble.tex
% !LPiL postamble = ./wtPostamble.tex

\lpilSection{fp-wt-integration}{Integration}
